documentclass{article}
renewcommandthesection{Roman{section}}
renewcommandthesubsection{thesection.roman{subsection}}
renewcommand{contentsname}{Table of Contents} 

usepackage[utf8]{inputenc}
usepackage{hyperref}
usepackage{graphicx}
usepackage{listings}
usepackage[usenames,dvipsnames]{xcolor}
usepackage{tikz}
usetikzlibrary{shapes.geometric, arrows}

lstset{frame=tb,
 backgroundcolor=color{white},   
    commentstyle=color{Gray},
    keywordstyle=color{orange},
    numberstyle=tinycolor{Gray},
    stringstyle=color{ForestGreen},
    basicstyle=ttfamilyfootnotesize,
    breakatwhitespace=false,         
    breaklines=true,                 
    captionpos=b,                    
    keepspaces=true,                 
    numbers=left,                    
    numbersep=5pt,                  
    showspaces=false,                
    showstringspaces=false,
    showtabs=false,                  
    tabsize=3}
tikzstyle{startstop} = [rectangle, rounded corners, minimum width=8cm, minimum height=1cm,text centered, draw=black, fill=white!30]
tikzstyle{io} = [rectangle, rounded corners, minimum width=8cm, minimum height=1cm,text centered, draw=black, fill=white!30]
tikzstyle{arrow} = [thick,-,=stealth]
title{DSA8002 Database and Programming Fundamentals Coursework Assignment Report}
author{Emir AkgiunStudent Number - 40367624}
date{}
begin{document}
maketitle
This assignment report contains a full walk-through of all the work process that took place in order to
complete the given tasks, supplemented by explanations, code snippets and auxiliary  graphics.
Other files that comprise the complete coursework submission and that may be referenced in this report can be found in the submitted ZIP archive. A more thorough and itemized description of the submission contents and dependencies could be obtained by consulting with the emph{README.txt} file, found in the exact same archive.

tableofcontents
section{Task 1}
In the first task the key to developing a proper solution was a thoughtful and mindful utilization
of emph{pandas} Data Frame objects and proper methods, allowing us to prepare, mutate and filter through
the given sets of data. Overall, the entire output pertaining to the task is contained within the emph{Task1.py} source code file.
subsection{Section A - Data Import}
As a matter of fact, the first subsection of the task required mostly basic methods found in the
emph{pandas}  package. First and foremost, it is detrimental that we  take a deeper look at the data we are provided with, so we can develop familiarity with it, make observations and potentially gain preliminary insight.
Initially, we are given 3 files with comma-separated values
begin{itemize}
  item emph{movies.csv} - key file containing all the data giving a basic description of movies in our database, including  their emph{movieId},emph{title},emph{genres}, separated by the pipe symbol and clearly having some genres written with lowercase first letter.
  item emph{ratings.csv} - a file containing ratings, which the users left to each movie emph{userId},emph{movieId},emph{rating},emph{timestamp}.
  item emph{tags.csv} - a file containing all the tags users assigned to each movie emph{userId},emph{movieId},emph{tag},emph{timestamp}.
end{itemize}

To begin with, the files should  be uploaded into data frames, which is easily done with the help of 
emph{pandas} methods. It is also  quite important to ensure we have some idea of what we have uploaded. For that purpose we could print the shapes of the new data frames, thus obtaining the number of data points and columns, as well as making sure  they have been uploaded correctly.

begin{lstlisting}[language=Python, caption=Task 1(A) - file importing]
         Section A - Import
        moviesDf = pd.read_csv(movies.csv)
        ratingsDf = pd.read_csv(ratings.csv)
        tagsDf = pd.read_csv(tags.csv)
        print(f{tagsDf.shape})
        print(f{ratingsDf.shape})
end{lstlisting}
Next up, in emph{Listings 2 and 3},  the pandas methods emph{df.dropduplicates()} and emph{df.dropna()} can be used to get rid of the duplicates and Na values in all data frames. While dropping Na values,
it is possible to  specify a subset of values, which, when repeated, result in duplicate rows being eliminated. In our case, user  can not leave two reviews for one movie, and one movieId can not have 
multiple titles, so in emph{Listing 2} we use sub-setting for the movies and ratings data frames.
begin{lstlisting}[language=Python, caption=Task 1(A) - duplicate removal]
        # Next up, using the dataframes created it is necessary to perform some  data cleaning
        # Dropping duplicates
        moviesDf = moviesDf.drop_duplicates(subset=[movieId, title])
        ratingsDf = ratingsDf.drop_duplicates(subset=[movieId, userId])
        tagsDf = tagsDf.drop_duplicates()
end{lstlisting}
begin{lstlisting}[language=Python, caption=Task 1(A) - Na values removal]
        moviesDf = moviesDf.dropna()
        ratingsDf = ratingsDf.dropna()
        tagsDf = tagsDf.dropna()
end{lstlisting}
It may be important to mention, that in the source code itself some small tests were also added for ensuring that the cleaning was  successful and there were no undesirable data points anymore.
The final preparatory step is to eliminate all data points in ratings and tags data frames, whose movieId
does not  belong to any movieId in the original movie data frame. Thus, only the relevant reviews and tags
are maintained.  For that  purpose the pandas emph{.loc()} method,which will allow for selection of specific rows, as well as emph{.isin()}, which compares the set of values to the value  of an object.
begin{lstlisting}[language=Python, caption=Task 1(A) - irrelevant  reviews and tags removal]
# Lastly, filtering out entries in Ratings, Tags  that  do NOT pertain to any entries in Movies
        allMovieId = moviesDf[movieId]
        ratingsDf = ratingsDf.loc[ratingsDf[movieId].isin(allMovieId)]
        tagsDf = tagsDf.loc[tagsDf[movieId].isin(allMovieId)]
end{lstlisting}
subsection{Section B - Similar  Movie Retrieval}
Moving on to the next section, we are tasked with finding similar movies to the one selected by the
user. As for now, it is necessary to retrieve all movies that have similar rating and at least one common genre.
Using a emph{while} loop the user is prompted to enter a desired movieId,  only proceeding if the movieId
exists. After that, I chose to save all the parameters relating to the chosen movie, such as its genres, title, ratings and tags.
begin{lstlisting}[language=Python, caption=Task 1(B) - User input prompting and saving parameters of the selected movie]
        allMovieId = allMovieId.tolist()
        selectedId = -1
        while (selectedId not in allMovieId) == True
                if test == False
                        print('Enter the Movie Id ')
                        selectedId = int(input(' '))
                else
                        print('Testing mode. Input predefined. ')
                        selectedId = int(testInput)
        # Saving the values of columns pertaining to the chosen movie
        selectedMovie = moviesDf['title'].loc[moviesDf['movieId'] == selectedId].values.tolist()
        selectedGenre = moviesDf['genres'].loc[moviesDf['movieId'] == selectedId].values
        selectedRatings = moviesDf['rating'].loc[moviesDf['movieId'] == selectedId].values
        selectedTags = moviesDf['tag'].loc[moviesDf['movieId'] == selectedId].values
end{lstlisting}
When working with any conventional data objects it is necessary to follow certain formal requirements.
So, in order for the merged data set to align with the First Normal Form, it is necessary  to use emph{pd.explode()} on that column and normalize the data set.
After that everything is ready for retrieving from the data frame using comparison operators and the emph{.isin()} method. The entire retrieval script can be see in the emph{Listing 6}.

begin{lstlisting}[language=Python, caption=Task 1(B) - Exploding genres column to conform to 1NF and retrieving a dataframe of similar movies]
selectedGenre = list(selectedGenre)[0]
        selectedRatings = list(selectedRatings)[0]
        mergedDf = mergedDf.explode('genres')
        mergedDf = mergedDf.drop_duplicates(subset=[genres, rating, tag])
        # Fetching movies with matching ratings and genres, dropping all the duplicate rows and  unnecessary columns
        similarMovies = mergedDf[mergedDf[movieId] != selectedId]
        similarMovies = similarMovies[similarMovies['genres'].isin(selectedGenre)&similarMovies['rating'].isin(selectedRatings)]
        similarMoviesFinal = similarMovies.drop_duplicates(subset = [movieId, title]).drop(labels = [rating, genres, tag],axis = 1)
        print(similarMoviesFinal)
end{lstlisting}
As a result, we obtain a data frame with movies, which contain at least  one common rating and genre and, 
usually, that is a pretty large number of similar movies. Luckily, the next task allows us to tweak the
retrieval algorithm, so the similarities found narrow down the resulting data frame even further.
subsection{Section C - Enhanced Similar Movie Retrieval}
In the third and final subsection of  Task 1 it is necessary to enhance (i.e. make narrower) the retrieval algorithm from subsection B. Frankly, there are many ways in which tag-based movie retrieval  could be implemented.
The algorithm implemented by myself for subsection C can be schematically summarized as follows

begin{tikzpicture}[node distance=1.5cm][Outline of the  tag-based retrieval algorithm]
node (start) [startstop] {Collect all tags left by users for every movie};
node (in1) [io, below of=start]{Remove duplicates and count tag frequency};
node (in2) [io, below of=in1]{Use these common tags to fetch similarly tagged movies};
draw [arrow] (start) -- (in1);
draw [arrow] (in1) -- (in2);
end{tikzpicture}

The few lines at the beginning of subsection C of Task1.py, actually, do the same filtering as in subsection B and create a proper list of tags from the selected movie. 
After that we should calculate the frequency of every tag appearing.It could be done by using the
pandas method emph{.size()} while grouping by movieId and tag. 

begin{lstlisting}[language=Python, caption=Task 1(С) - tag frequency calculation]
        # Creating a dataframe containing the number of instances of each tag being  used on every movie
        similarMoviesFreq = similarMoviesEnhanced.groupby(['movieId','tag']).size()
        #If for some reason no dataframe is created, it is necessary to change that object's type
        if not isinstance(similarMoviesFreq, pd.DataFrame)
                similarMoviesFreq = similarMoviesFreq.to_frame()
end{lstlisting}
Then code below mutates the frequencies data frame so it can be joined in with the movies data frame later on.
begin{lstlisting}[language=Python, caption=Task 1(С) - frequency data frame preparation]
        # Mutating the dataframes in an appropriate way
        similarMoviesFreq = similarMoviesFreq.reset_index()
        similarMoviesFreq = similarMoviesFreq.rename(columns={0 'tag frequency'})
        similarMoviesEnhanced = similarMoviesEnhanced.drop_duplicates(subset=[tag]).drop(labels = [rating, genres],axis = 1)
        similarMoviesEnhanced =  similarMoviesEnhanced.reset_index()
end{lstlisting}
Finally, we can find other movies that have similar tags to the ones we have in the similarMoviesFreq data frame. Movies are also sorted by their similar tag frequency. However, if no movie fitting all the criteria can be fetched  (i.e. returning an empty data frame), the conditional statement  prints out a relevant prompt and sets the value of similarMoviesEnhancedFinal to None. this variable is returned by function, however, previously in Task 2 a table of similar movies was still printed out and could be used to identify more or less similar movies anyway.
In case the search was a success, the function Task1() returns a filled data frame and prints it, dropping all columns except movieId and movie titles.
begin{lstlisting}[language=Python, caption=Task 1(С) - finding similar tags]
        # Checking whether any similar tag frequencies were added and append them to the dataframesort the frequencies
        if not similarMoviesFreq.empty
                similarMoviesEnhanced = similarMoviesEnhanced.join(similarMoviesFreq['tag frequency'])
                similarMoviesEnhanced= similarMoviesEnhanced.sort_values('tag frequency', ascending=False)
        # Final output of the function, the tag-coincidence rate and a skimmed best-matching movies dataframe
        if similarMoviesEnhanced.empty
                print(fNo movies of similar genre and rating share same tags as the movie you chose ( )
                similarMoviesEnhancedFinal  = None
        else
                print(fHere are the matching movies of the same genre(s) and similar ratings, sorted  by common tag frequency )
                print(similarMoviesEnhanced)
                print(fFinally, just the best matching movies )
                similarMoviesEnhancedFinal = similarMoviesEnhanced.drop_duplicates(subset=[movieId, title]).drop(labels=[tag, tag frequency,index], axis=1)
                print(similarMoviesEnhancedFinal)
        return similarMoviesEnhancedFinal
end{lstlisting}
That snippet finalizes the entire task, returning a variable emph{similarMoviesEnhancedFinal} in case tag-based similarities are encountered, which is a data frame with  movie titles and movieIds.
Anyhow, below the Task1() function body and its call there is a chunk of code that performs a
functional test of the task function. More on that test could be found in the hyperref[sectask4]{emph{Task4}}
section.
section{Task 2}
In the following task the main goal is to create a series of SQL statements so as when
they are executed, they create tables equal to data frames used previously. The only exception is that 
a separate table for genres should be created as well, as storing it within the movies table is inappropriate and violates 1NF. Also, the string values in data frames should be wrapped in quotation marks so they are passed properly into the mask and do not create any SQL injections e.g. SQL reserved keyword ACTION could be ambiguous with Action as a genre.
For the purposes of generation of a  bunch of working SQL script files, Python should come in handy.

begin{lstlisting}[language=Python, caption=Task 2 - Python script for generating SQL statements]
               def createSQL(dataFrame, tableName)
                # Obtaining the list of columns and data types  in the dataframe
                columns = []
                for col, dtype in zip(dataFrame.columns, dataFrame.dtypes)
                        if str(dtype) == 'object'
                                dtype = 'VARCHAR(255)'
                        elif 'int' in str(dtype)
                                dtype = 'INT'
                        elif 'float' in str(dtype)
                                dtype = 'FLOAT'
                        elif 'datetime' in str(dtype)
                                dtype = 'DATETIME'
                        # Creating a list of column names and the data types of values in it
                        columns.append(f{col} {dtype})
                # Creating the table
                columnStr = , .join(columns)
                createTable = fCREATE TABLE {tableName} ({columnStr});
                #  Finally, we can write the created strings into a .sql file
                with open(f'{tableName}.sql', 'wb') as f
                        # Write the CREATE TABLE statement
                        f.write(createTable.encode('utf-8'))
                        # The string is binary to facilitate proper writing to the document
                        f.write(b'n')
                        # Building and writing the INSERT INTO statements, looping  through the dataframe
                        for _, row in dataFrame.iterrows()
                                insert = fINSERT INTO {tableName}({', '.join(dataFrame.columns)}) VALUES ({', '.join([str(x) for x in row])});
                                f.write(insert.encode('utf-8'))
                                f.write(b'n')
end{lstlisting}
The code above first checks data types in columns in the data frame that was passed as argument to the function emph{createSQL} and creates a list consisting of column names and data type, but written in a SQL style. Then it creates and writes the CREATE TABLE statement with all the previously created values of column names and data types. Lastly, iterating on all the rows in the argument data frame, the function writes INSERT INTO statements according to the mask.
Below is the example of a few lines from the newly generated movies.sql script. 
begin{lstlisting}[language=SQL, caption=Task 2 - Generated SQL statements example movies.sql]
       CREATE TABLE movies (movieId INT, title VARCHAR(255));
INSERT INTO movies(movieId, title) VALUES (1, Toy Story (1995));
INSERT INTO movies(movieId, title) VALUES (2, Jumanji (1995));
INSERT INTO movies(movieId, title) VALUES (3, Grumpier Old Men (1995));
INSERT INTO movies(movieId, title) VALUES (4, Waiting to Exhale (1995));
INSERT INTO movies(movieId, title) VALUES (5, Father of the Bride Part II (1995));
end{lstlisting}
section{Task 3}
To begin with, Task 3 is quite similar to Task 1 in its gist, with the only large difference being 
that in this case it is necessary to use SQL queries to retrieve similar movies. emph{Pandasql} package 
is used for that  purpose, so as to allow  for operating over existing data frames.
The general algorithm of cleaning and preparing data in Task 3 is, as can be observed in the source code files, almost identical to the same 
subsection{Section A - Similar  Movie Retrieval with SQL}
Importantly, it could be noticed that the SQL queries are formatted strings in my source code. Making  the string
formatted allows us to insert variablesfunctions from Python into SQL statements.
In the query below, we assign 
begin{lstlisting}[language=SQL, caption=Task 3 - SQL over dataframes to fetch similar movies]
SELECT DISTINCT m.movieId, m.title 
FROM moviesDf m 
JOIN ratingsDf r ON m.movieId = r.movieId 
JOIN genresDf g ON m.movieId =  g.movieId 
WHERE r.rating IN (SELECT DISTINCT rating from ratingsDf r WHERE r.movieId = {selectedId}) 
AND g.genres IN (SELECT DISTINCT genres  FROM genresDf g WHERE g.movieId = {selectedId}) 
AND m.movieId != {selectedId};
end{lstlisting}
subsection{Section B - Enhanced Similar  Movie Retrieval with SQL}
In order to add tagsDf into the mix it is necessary to add one more JOIN statement and add 1 more AND to the WHERE clause. In result, the SQL query below outputs a more concise return, while getting rid of movies with no similar tags to the selection.
begin{lstlisting}[language=SQL, caption=Task 3 - SQL over dataframes to fetch similar movies with tags]
SELECT DISTINCT m.movieId, m.title
FROM moviesDf m 
JOIN ratingsDf r ON m.movieId = r.movieId 
JOIN genresDf g ON m.movieId =  g.movieId 
JOIN tagsDf t ON m.movieId = t.movieId 
WHERE r.rating IN (SELECT DISTINCT rating from ratingsDf r WHERE r.movieId = {selectedId}) 
AND g.genres IN (SELECT DISTINCT genres  FROM genresDf g WHERE g.movieId = {selectedId}) 
AND t.tag IN (SELECT DISTINCT tag  FROM tagsDf t WHERE t.movieId = {selectedId}) 
AND m.movieId != {selectedId};
end{lstlisting}
That returns a data frame with all data points satisfying the conditions and ,thus, sums up Task 3. There is also a test chunk of code (which is covered in Task 4 section) below main task code in the same file.
section{Task 4}
label{sectask4}
As was previously mentioned, I made a decision to implement functional tests at the end of every  task,
which I personally found to be a more logical approach as long as the tasks are evaluated separately and sequentially.
However, in the source code file emph{Task4.py} all the code snippets are combined for the purpose of  demonstration of the tests implemented for the fourth. In tasks 1 and 3, the functions containing all 
operations 
subsection{Test  Case 1}
As for the first  task, I developed a test case where the theoretical user tries to find a film
similar to the The Right Stuff by inputting a movieId. In this case, Task1() function is passed
arguments (test = True, inputValue = 1231). That disables the user input solicitation and the function uses the inputValue as the movieId.
The variable  emph{expectedOutput} is also filled manually. The test checks whether the return of the function Task1() is identical to the expected output. The test should return successfuul if the output 
of the funtion and the expected return are equal.
begin{lstlisting}[language=Python, caption=Code for Test Case 1]
 Test for Task 1 
Test Case 1 A user tries to find a similar movie to the one with ID 1231 The Right Stuff. The output should contain a movie titled Apollo 13, which has an ID of 150
from Task1 import Task1
testValue = 1231
expectedOutput = pd.DataFrame({movieId 150, title Apollo 13 (1995)}, index=[0])
testOutput =  Task1(test = True, testInput = testValue)
if testOutput.equals(expectedOutput)
        print(Test 1 passed successfully.)
else
        print(Test 1 failed.)
end{lstlisting}
subsection{Test Case 2}
As for the second  task, I developed a test case to check whether the functionality of the script generating the SQL statements is working properly and builds correct tables.
In order to actually build a database, a module emph{sqlite3} found in standard python library is used.
After creating a local  connection to a database file, it is possible to execute SQL in it. For the sake
of the test, I only run SQL files containing CREATE TABLE and INSERT  INTO statements in movies.sql and genres.sql, so it is possible to execute a query in the  question.
When this test is run in the Task2.py script, the test should be successful.
begin{lstlisting}[language=Python, caption=Code for Test Case 2 including  commented out database creation block]
 Test for Task 2 
Test Case 2 A database developer wants to fetch movie title and genre with moviedId set to 2  
#It is necessary to  set up a local database in order to create SQL tables with our files
conn = sqlite3.connect(testdatabase.db)
cursor = conn.cursor()
#Executing our generated SQL files. !!! IF DATABASE ALREADY EXISTS, COMMENT OUT THIS SECTION
# with open('movies.sql', 'r') as f
#         sql1 = f.read()
#         cursor.executescript(sql1)
# with open('genres.sql', 'r') as f
#         sql2 = f.read()
#         cursor.executescript(sql2)
# Commiting the changes to the database
conn.commit()
#END OF DATABASE CREATION!!!

# Selecting all data needed on movieId 2 into a dataframe
testMoviesDf = pd.read_sql(SELECT title, genres FROM movies, genres WHERE movies.movieId == 2 AND genres.movieId == 2, conn)
print(testMoviesDf.head())
expectedOutput = pd.DataFrame({title ['Jumanji (1995)','Jumanji (1995)','Jumanji (1995)'], genres [adventure, children, fantasy]}, index=[0,1,2])
# Comparing to expected behaviour
if testMoviesDf.equals(expectedOutput)
        print(Test 2 passed successfully.)
else
        print(Test 2 failed.)
end{lstlisting}
subsection{Test Case 3}
As for the third  task, I developed a test case that is similar to  the one in Task 1.
begin{lstlisting}[language=Python, caption= Code for Test Case 3]
 Test for Task 3 
Test Case 3 A user aims to find a similar movie to the one with ID 2. 
expectedOutput = pd.DataFrame({movieId [4993, 7153, 46972, 59501, 80834, 106489], title [Lord of the Rings The Fellowship of the Ring, The (2001),Lord of the Rings The Return of the King, The (2003),Night at the Museum (2006),Chronicles of Narnia Prince Caspian, The (2008), Sintel (2010), Hobbit The Desolation of Smaug, The (2013)]}, index=[0,1,2,3,4,5])
testOutput = Task3(test = True, testInput= 2)
if testOutput.equals(expectedOutput)
        print(Test 3 passed successfully.)
else
        print(Test 3 failed.)
end{lstlisting}
 
That concludes this report on the assignment. All the tests should pass successfully as long as they run at the end of their respective task's source  code. Overall, we successfully managed to prepare and wrangle the provided data, perform necessary retrieval and  filtering tasks as well as to build  SQL script files that create SQL tables from data frames and to test the functionally of the source code.

end{document}